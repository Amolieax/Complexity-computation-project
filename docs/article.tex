\documentclass[conference]{IEEEtran}
\usepackage{graphicx}
\usepackage[utf8]{inputenc}
\usepackage[T2A]{fontenc}
\usepackage[english,main=russian]{babel}
\usepackage{csquotes}
\usepackage{amsfonts}
\usepackage{hyperref}
\usepackage{minted}
\usepackage[style=numeric]{biblatex}
\addbibresource{reference.bib}

\usepackage{geometry, microtype, fancyhdr, titlesec, xcolor, graphicx}

\geometry{a4paper, left=2.5cm, right=2.5cm, top=2.5cm, bottom=2.5cm}
\definecolor{accent}{RGB}{45, 90, 150}
\titleformat{\section}{\color{accent}\Large\bfseries\sffamily}{\thesection}{0.5em}{}

\begin{document}

\title{Реализация и анализ производительности алгоритмов для (1, 2)-задачи комивояжёра.}
\author{Миатов Александр. Б05-327}
\maketitle

\begin{abstract}
    В статье «8/7-approximation algorithm for (1,2)-TSP» авторства П. Бермана и М. Карпинского \cite{Berman_2006} представлен алгоритм приближённого решения (1,2)-задачи коммивояжёра с весами рёбер 1 или 2, гарантирующий коэффициент приближения \(\frac{8}{7}\). В данной работе реализован этот алгоритм, проверена его корректность на тестовых примерах и проведён сравнительный анализ времени работы и качества решения на графах различной структуры и размера.
\end{abstract}

\section{Введение}
(1,2)-задача коммивояжёра — это частный случай метрической задачи коммивояжёра, где все веса рёбер принимают значения только 1 или 2. Задача остаётся \textbf{NP-трудной}, но допускает лучшие приближённые алгоритмы по сравнению с общей метрической задачей. 

Алгоритм Бермана–Карпинского, представленный в 2006 году, достигает коэффициента приближения \(\frac{8}{7}\), что улучшает предыдущий результат \(\frac{7}{6}\) Пападимитриу–Яннакакиса \cite{Papadimitriou_1993}. В основе алгоритма лежит комбинация методов построения остовного дерева, поиска паросочетаний и анализа специальных структур графа.

В данной статье реализован алгоритм Бермана–Карпинского, протестирована его эффективность на различных типах графов и проведено сравнение с теоретическими оценками. Особое внимание уделено анализу поведения алгоритма на графах с разным распределением весов 1 и 2.

\end{document}